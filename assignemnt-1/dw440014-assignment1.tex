\documentclass[12pt]{article}
\usepackage[utf8]{inputenc}
\usepackage[T1]{fontenc}
\usepackage{indentfirst}
\usepackage{amsmath}
\usepackage{amssymb}
\usepackage{natbib}
\usepackage{graphicx}
\usepackage{float}
\usepackage[a4paper, margin = 2 cm]{geometry}
\usepackage{fancyhdr}
\usepackage{wrapfig}
\usepackage{hyperref}
\usepackage{mathtools}
\usepackage{algorithm}
\usepackage{algpseudocode}

\title{Selected Topics in Graph Theory Assignment 1}
\author{Dominik Wawszczak}
\date{2025-03-18}

\begin{document}
	\setlength{\parindent}{0 cm}
	
	Dominik Wawszczak \hfill Selected Topics in Graph Theory
	
	Student ID Number: 440014 \hfill Assignment 1
	
	Group Number: 2
	
	\bigskip
	\hrule
	\bigskip
	
	\textbf{Problem 2}
	
	\medskip
	
	Consider a bipartite graph \(G = (X, C, E)\), where \(X = \{x_{1}, x_{2},
	\ldots, x_{n}\}\), \(C = \{C_{1}, C_{2}, \ldots, C_{m}\}\), and \(E =
	\{(x_{i}, C_{j}) \ : \ C_{j} \ \text{contains either} \ x_{i} \ \text{or} \
	\neg x_{i}\}\). This graph is \(3\)-regular, as each variable appears in
	exactly three clauses, and every clause contains exactly three literals
	corresponding to pairwise distinct variables. A direct consequence of this
	observation is that \(n = m\), since \(3n = |E| = 3m\).
	
	\medskip
	
	We will now prove that Hall’s condition holds in \(G\). Let \(S\) be any
	subset of \(X\). Define \(E_{S} = \{(x_{i}, C_{j}) \ : \ (x_{i}, C_{j}) \in
	E \ \wedge \ x_{i} \in S\}\) as the set of edges between \(S\) and
	\(N_{G}(S)\). Since \(|E_{S}| = 3 |S|\), we obtain \(3 |S| \leqslant 3
	|N_{G}(S)|\), confirming that \(G\) satisfies Hall’s condition. Therefore,
	\(G\) has a perfect matching.
	
	\medskip
	
	We can satisfy each clause \(C_{j}\) by assigning the appropriate truth
	value to its matched variable. This ensures that the entire formula is
	satisfied, completing the proof.
\end{document}
