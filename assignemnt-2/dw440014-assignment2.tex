\documentclass[12pt]{article}
\usepackage[utf8]{inputenc}
\usepackage[T1]{fontenc}
\usepackage{indentfirst}
\usepackage{amsmath}
\usepackage{amssymb}
\usepackage{natbib}
\usepackage{graphicx}
\usepackage{float}
\usepackage[a4paper, margin = 2 cm]{geometry}
\usepackage{fancyhdr}
\usepackage{wrapfig}
\usepackage{hyperref}
\usepackage{mathtools}
\usepackage{algorithm}
\usepackage{algpseudocode}

\title{Selected Topics in Graph Theory Assignment 2}
\author{Dominik Wawszczak}
\date{2025-05-18}

\begin{document}
	\setlength{\parindent}{0 cm}
	
	Dominik Wawszczak \hfill Selected Topics in Graph Theory
	
	Student ID Number: 440014 \hfill Assignment 2
	
	Group Number: 2
	
	\bigskip
	\hrule
	\bigskip
	
	\textbf{Problem 4}
	
	\medskip
	
	Let \(S \subseteq V(G)\) be an independent set of \(G\), and define a vector
	\(x \in \mathbb{R}^n\) such that \(x_{i} = 1\) if \(i \in S\), and \(x_{i} =
	0\) otherwise. Decompose \(x\) as
	\[
		x \ = \ \frac{|S|}{n} \cdot \mathbf{1} + y \text{.}
	\]
	Then \(y\) is orthogonal to \(\mathbf{1}\), since
	\[
		y^{T} \mathbf{1} \ = \ \bigg( x - \frac{|S|}{n} \mathbf{1} \bigg)^{T}
		\mathbf{1} \ = \ x^{T} \mathbf{1} - \frac{|S|}{n} \mathbf{1}^{T}
		\mathbf{1} \ = \ |S| - \frac{|S|}{n} n \ = \ 0 \text{.}
	\]
	
	\medskip
	
	Let \(A\) be the adjacency matrix of \(G\). Since \(S\) is an independent
	set, it follows that
	\[
		x^{T} A x \ = \ \sum\limits_{i \in S} \sum\limits_{j \in S} A_{ij} \ = \
		0 \text{,}
	\]
	Expanding \(x\) using the decomposition, we get
	\[
		0 \ = \ x^{T} A x \ = \ \bigg( \frac{|S|}{n} \mathbf{1} + y \bigg)^{T} A
		\bigg( \frac{|S|}{n} \mathbf{1} + y \bigg) \ = \ \frac{|S|^{2}}{n^{2}}
		\mathbf{1}^{T} A \mathbf{1} + 2 \frac{|S|}{n} \mathbf{1}^{T} A y + y^{T}
		A y \text{.}
	\]
	Since \(\mathbf{1}\) is an eigenvector of \(A\) with eigenvalue \(d\), we
	have \(A \mathbf{1} = d \mathbf{1}\), and thus
	\[
		\frac{|S|^{2}}{n^{2}} \mathbf{1}^{T} A \mathbf{1} \ = \
		\frac{|S|^{2}}{n^{2}} \mathbf{1}^{T} d \mathbf{1} \ = \
		\frac{|S|^{2}}{n^{2}} d n \ = \ \frac{|S|^{2} d}{n} \text{.}
	\]
	Also,
	\[
		2 \frac{|S|}{n} \mathbf{1}^{T} A y \ = \ 2 \frac{|S|}{n} \mathbf{1}^{T}
		d y \ = \ 0 \text{,}
	\]
	since \(\mathbf{1}^{T} y = 0\) by orthogonality. Therefore,
	\[
		0 \ = \ \frac{|S|^{2} d}{n} + y^{T} A y \text{.}
	\]
	
	\medskip
	
	Rearranging gives
	\[
		- \frac{|S|^{2} d}{n} \ = \ y^{T} A y \ \geqslant \ \lambda_{n} \cdot
		\|y\|^{2} \text{.}
	\]
	We compute
	\[
		\|y\|^{2} \ = \ \bigg\| x - \frac{|S|}{n} \mathbf{1} \bigg\|^{2} \ = \
		|S| - 2 \frac{|S|^{2}}{n} + \frac{|S|^{2}}{n} \ = \ |S| -
		\frac{|S|^{2}}{n} \text{,}
	\]
	so
	\[
		- \frac{|S|^{2} d}{n} \ \geqslant \ \lambda_{n} \cdot \bigg( |S| -
		\frac{|S|^{2}}{n} \bigg) \text{.}
	\]
	Dividing both sides by \(|S|\) yields
	\[
		- \frac{|S| d}{n} \ \geqslant \ \lambda_{n} \cdot \bigg( 1 -
		\frac{|S|}{n} \bigg) \quad \implies \quad |S| \cdot d \ \leqslant \
		\lambda_{n} |S| - \lambda_{n} \cdot n \quad \implies \quad |S| \
		\leqslant \ n \cdot \frac{- \lambda_{n}}{d - \lambda_{n}} \text{,}
	\]
	which completes the proof.
\end{document}
