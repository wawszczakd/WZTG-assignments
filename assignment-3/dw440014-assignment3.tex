\documentclass[12pt]{article}
\usepackage[utf8]{inputenc}
\usepackage[T1]{fontenc}
\usepackage{indentfirst}
\usepackage{amsmath}
\usepackage{amssymb}
\usepackage{natbib}
\usepackage{graphicx}
\usepackage{float}
\usepackage[a4paper, margin = 2 cm]{geometry}
\usepackage{fancyhdr}
\usepackage{wrapfig}
\usepackage{hyperref}
\usepackage{mathtools}
\usepackage{algorithm}
\usepackage{algpseudocode}
\usepackage{enumitem}

\title{Selected Topics in Graph Theory Assignment 3}
\author{Dominik Wawszczak}
\date{2025-06-14}

\begin{document}
	\setlength{\parindent}{0 cm}
	
	Dominik Wawszczak \hfill Selected Topics in Graph Theory
	
	Student ID Number: 440014 \hfill Assignment 3
	
	Group Number: 2
	
	\bigskip
	\hrule
	\bigskip
	
	\textbf{Problem 1}
	
	\medskip
	
	Let \(G\) be a simple planar graph with \(v\) vertices, \(e\) edges, and
	\(f\) faces. Suppose \(G\) has \(t\) triangular faces and that every vertex
	has degree at least \(5\). Since the minimum degree is \(5\), we have \(2 e
	\geqslant 5 v\). By Euler's formula:
	\[
		v - e + f \ = \ 2 \quad \implies \quad f - 2 \ = \ e - v \ \geqslant \
		\frac{3}{5} e \text{.}
	\]
	Let \(f_{i}\) denote the number of faces of size \(i\). Then:
	\[
		2 e \ = \ \sum\limits_{i \geqslant 3} i f_{i} \ = \ 3 t +
		\sum\limits_{i \geqslant 4} i f_{i} \ \geqslant \ 3 t + 4 \cdot
		\sum\limits_{i \geqslant 4} f_{i} \ = \ 3 t + 4 (f - t) \ = \ 4 f - t
		\text{.}
	\]
	Thus:
	\[
		t \ \geqslant \ 4 f - 2 e \ \geqslant \ 4 f - \frac{10}{3} \cdot (f - 2)
		\ = \ \frac{2 f + 20}{3} \ \geqslant \ \frac{2 t + 20}{3} \text{,}
	\]
	which implies:
	\[
		3 t \ \geqslant \ 2 t + 20 \quad \implies \quad t \ \geqslant \ 20
		\text{.}
	\]
	Hence, \(k \geqslant 19\). Since the icosahedral graph is a \(5\)-regular
	simple planar graph with exactly \(20\) triangular faces, \(k = 19\).
	
	\bigskip
	
	\textbf{Problem 2}
	
	\medskip
	
	For any \(i \in \{1, 2, \ldots, k\}\) and any \(u\) in the same connected
	component as \(v_{i}\), we have
	\[
		\bigg| p_{t}(v_{i}, u) - \frac{1}{n} \bigg| \ \leqslant \
		(1 - \alpha)^{t} \text{,}
	\]
	where \(p_{t}(a, b)\) is the probability that a random walk of length \(t\)
	starting at \(a\) ends at \(b\). By setting \(t = \Big\lceil
	\frac{\ln (2 n)}{- \ln (1 - \alpha)} \Big\rceil = \mathcal{O}(\log n)\), we
	obtain
	\[
		\bigg| p_{t}(v_{i}, u) - \frac{1}{n} \bigg| \ \leqslant \ \frac{1}{2 n}
		\text{,} \quad \text{and therefore} \quad p_{t}(v_{i}, u) \in \bigg[
		\frac{1}{2 n}, \frac{3}{2 n} \bigg] \text{.}
	\]
	
	\medskip
	
	If we perform \(m\) such walks, the probability that at least one ends at
	\(u\) is at least
	\[
		1 - \bigg( 1 - \frac{1}{2n} \bigg)^{m} \text{.}
	\]
	We want this probability to be at least \(\frac{1}{\sqrt{n}}\). Since
	\[
		\bigg( 1 - \frac{1}{2 n} \bigg)^{m} \ \leqslant \ e^{- \frac{m}{2 n}}
		\text{,}
	\]
	it suffices that \(m\) satisfies
	\[
		e^{- \frac{m}{2 n}} \ \leqslant \ 1 - \frac{1}{\sqrt{n}} \ = \
		e^{\ln \big( 1 - \frac{1}{\sqrt{n}} \big)} \quad \implies \quad -
		\frac{m}{2 n} \ \leqslant \ \ln \bigg( 1 - \frac{1}{\sqrt{n}} \bigg)
		\text{,}
	\]
	hence
	\[
		m \ \geqslant \ - 2 n \ln \bigg( 1 - \frac{1}{\sqrt{n}} \bigg) \ = \ 2 n
		\cdot \sum\limits_{i = 1}^{\infty}
		\frac{\Big( \frac{1}{\sqrt{n}} \Big)^{i}}{i} \ = \ 2 \sqrt{n} + 1 + 2
		\cdot \sum\limits_{i = 3}^{\infty}
		\frac{\Big( \frac{1}{\sqrt{n}} \Big)^{i - 2}}{i} \text{.}
	\]
	It is enough to set
	\begin{align*}
		m \ &= \ \Bigg\lceil 2 \sqrt{n} + 1 + 2 \cdot
		\sum\limits_{i = 3}^{\infty} \bigg( \frac{1}{\sqrt{n}} \bigg)^{i - 2}
		\Bigg\rceil \ = \ \Bigg\lceil 2 \sqrt{n} + 1 + \frac{2}{\sqrt{n}} \cdot
		\sum\limits_{i = 0}^{\infty} \bigg( \frac{1}{\sqrt{n}} \bigg)^{i}
		\Bigg\rceil \ = \\
		&= \ \Bigg\lceil 2 \sqrt{n} + 1 + \frac{2}{\sqrt{n}} \cdot
		\frac{1}{1 - \frac{1}{\sqrt{n}}} \Bigg\rceil \ = \ \bigg\lceil 2
		\sqrt{n} + 1 + \frac{2}{\sqrt{n} - 1} \bigg\rceil \text{,}
	\end{align*}
	so \(m = \mathcal{O} \big( \sqrt{n} \big)\).
	
	\medskip
	
	At initialization, for each \(i \in \{1, 2, \ldots, k\}\), we perform \(m\)
	random walks of length \(t\) from \(v_{i}\), and construct a set \(R_{i}\)
	containing all reached vertices. This ensures that if \(u\) is in the same
	component as \(v_{i}\), then \(u \in R_{i}\) with probability at least
	\(\frac{1}{\sqrt{n}}\). The total time complexity of initialization is
	\(\mathcal{O} \big( k \sqrt{n} \log n \big)\).
	
	\medskip
	
	To answer a query, we perform \(m\) random walks of length \(t\) from \(x\),
	and create a set \(Q\) of all reached vertices. We then search for \(i \in
	\{1, 2, \ldots, k\}\) such that \(Q \cap R_{i} \neq \emptyset\). The time
	complexity of a single query is \(\mathcal{O} \big( k \sqrt{n} \log n
	\big)\), as it can be checked whether \(Q \cap R_{i} \neq \emptyset\) in
	time \(\mathcal{O} ((|Q| + |R_{i}|) \log (|Q| + |R_{i}|)) = \mathcal{O}
	\big( \sqrt{n} \log n \big)\).
	
	\medskip
	
	It remains to estimate the probability that \(Q \cap R_{i} \neq \emptyset\),
	where \(i\) is such that \(x\) and \(v_{i}\) lie in the same connected
	component. Since for any \(u\) in this component, the probabilities that \(u
	\in R_{i}\) and \(u \in Q\) are both at least \(\frac{1}{\sqrt{n}}\), the
	probability that \(u\) belongs to both these sets is at least
	\(\frac{1}{n}\). Thus, the probability that no such \(u\) exists is at most
	\[
		\bigg( 1 - \frac{1}{n} \bigg)^{n} \ \leqslant \ e^{-1} \text{,}
	\]
	and so the probability that \(Q \cap R_{i} \neq \emptyset\) is at least
	\(1 - e^{-1} \geqslant \frac{1}{2}\).
\end{document}
