\documentclass[12pt]{article}
\usepackage[utf8]{inputenc}
\usepackage[T1]{fontenc}
\usepackage{indentfirst}
\usepackage{amsmath}
\usepackage{amssymb}
\usepackage{natbib}
\usepackage{graphicx}
\usepackage{float}
\usepackage[a4paper, margin = 2 cm]{geometry}
\usepackage{fancyhdr}
\usepackage{wrapfig}
\usepackage{hyperref}
\usepackage{mathtools}
\usepackage{algorithm}
\usepackage{algpseudocode}
\usepackage{enumitem}

\title{Selected Topics in Graph Theory Assignment 3}
\author{Dominik Wawszczak}
\date{2025-06-14}

\begin{document}
	\setlength{\parindent}{0 cm}
	
	Dominik Wawszczak \hfill Selected Topics in Graph Theory
	
	Student ID Number: 440014 \hfill Assignment 3
	
	Group Number: 2
	
	\bigskip
	\hrule
	\bigskip
	
	\textbf{Problem 1}
	
	\medskip
	
	Let \(G\) be a simple planar graph with \(v\) vertices, \(e\) edges, and
	\(f\) faces. Suppose \(G\) has \(t\) triangular faces and that every vertex
	has degree at least \(5\). Since the minimum degree is \(5\), we have \(2 e
	\geqslant 5 v\). By Euler's formula:
	\[
		v - e + f \ = \ 2 \quad \implies \quad f - 2 \ = \ e - v \ \geqslant \
		\frac{3}{5} e \text{.}
	\]
	Let \(f_{i}\) denote the number of faces of size \(i\). Then:
	\[
		2 e \ = \ \sum\limits_{i \geqslant 3} i f_{i} \ = \ 3 t +
		\sum\limits_{i \geqslant 4} i f_{i} \ \geqslant \ 3 t + 4 \cdot
		\sum\limits_{i \geqslant 4} f_{i} \ = \ 3 t + 4 (f - t) \ = \ 4 f - t
		\text{.}
	\]
	Thus:
	\[
		t \ \geqslant \ 4 f - 2 e \ \geqslant \ 4 f - \frac{10}{3} \cdot (f - 2)
		\ = \ \frac{2 f + 20}{3} \ \geqslant \ \frac{2 t + 20}{3} \text{,}
	\]
	which implies:
	\[
		3 t \ \geqslant \ 2 t + 20 \quad \implies \quad t \ \geqslant \ 20
		\text{.}
	\]
	Hence, \(k \geqslant 19\). Since the icosahedral graph is a \(5\)-regular
	simple planar graph with exactly \(20\) triangular faces, \(k = 19\).
\end{document}
